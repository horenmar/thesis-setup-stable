% opmac-bib
%%%%%%%%%%%%%%%%%%%%%%%%%%%%%%%%%%%%%%%%%%%%%%%%%%
% Petr Olsak, version Apr. 2015 (beta)

% This is module for OPmac macros, see http://petr.olsak.net/opmac.html
% The command:
%    \usebib/<sorttype> (<style>) (bibfiles>
% for direct reading .bib files by OPmac macros (without bibTeX) is implemented here.
% The implementation depends on librarian.tex package by Paul Isambert.
% Use:
%    \input opmac  \input opmac-bib
% or simply:
%    \input opmac-bib
% before using the command \usebib.
% For more information see the end of this file.

\ifx\usebib\udefined \else \endinput \fi
\ifx\OPmacversion\undefined \input opmac \fi
\ifx\printbib\undefined 
   \errmessage{Please, upgrade OPmac to version Apr. 2014 or newer}
   \endinput\fi

% we needn't \errmessage when bad TeX engnine is detected during \input librarian:

\def\tmp{}
\let\errmessageori=\errmessage
\def\errmessage#1{\def\tmp{error}}
\input librarian
\let\errmessage=\errmessageori
\ifx\tmp\empty\else
  \def\usebib/#1 (#2) #3 {%
     \opwarning{eTeX and (pdfTeX or XeTeX or LuaTeX) not detected}%
     \immediate\write16{\space\space
               But librarian package needs it. \noexpand\usebib ignored.}%
  }
  \endinput \fi

% The \usebib command:

\def\usebib/#1 (#2) #3 {%
  \ifx\citelist\empty
     \opwarning{No cited items. \noexpand\usebib ignored}%
  \else
     \bgroup \par
        \ifx\bibpart\undefined \def\bibpart{none}\fi
        \catcode`<=12
        \ifx\savedttchar\undefined\else \catcode\savedttchar=12 \fi
        \def\opmacbibstyle{#2}%
        \input opmac-bib-#2
        \bibtexhook
        \let\citeI=\relax \xdef\citelist{\citelist\citelistB}%
        \global\let\addcitelist=\writeXcite
        \def\tmp##1[*]##2\relax{\def\tmp{##2}}\expandafter\tmp\citelist[*]\relax
        \ifx\tmp\empty\else % there was \nocite[*] used. 
           \setbox0=\vbox{\def\citelist{}\adef@{\readentry}%
           \input #3.bib
           \expandafter}\expandafter\def\expandafter\citelist\expandafter{\citelist}%
        \fi
        \def\citeI[##1]{\csname lb@cite\endcsname{##1}{\bibpart}{}{}}\citelist
        \BibFile{#3}%
        \if s#1\def\warntwobytes{}\def\warnthreebytes{}% \scantokens is used during sorting
           \SortList{\bibpart}\fi
        \ReadList{\bibpart}%
     \egroup
  \fi
}
\def\readentry#1#{\readentryA}
\def\readentryA#1{\readentryB#1,,\relax!.}
\def\readentryB#1#2,#3\relax!.{\addto\citelist{\citeI[#1#2]}}

% Corrections in librarian macros:

\tmpnum=\catcode`\@  \catcode`\@=11
\def\lb@checkmissingentries#1,{% we needn't \errmessage here, only \opmacwarning
  \def\lb@temp{#1}%
  \unless\ifx\lb@temp\lb@eoe
    \lb@ifcs{#1}{fields}%
            {}%
            {\opwarning{\string\usebib: entry [#1] isn't found in .bib file(s)}}%
    \expandafter\lb@checkmissingentries
  \fi
}
\def\lb@readentry#1#2#3,{% space before key have to be ingnored
  \def\lb@temp{#2#3}%      we need case sensitive keys
  \lb@ifcs\lb@temp{requested}%
          {\let\lb@entrykey\lb@temp
           \lb@defcs\lb@entrykey{fields}{}%
           \lowercase{\lb@addfield{entrytype}{#1}}%
           \let\lb@next\lb@analyzeentry}%
          {\def\lb@next{\expandafter\lb@gotoat\lb@gobbletoeoe}}%
  \lb@next
}
\let\lb@compareA=\lb@compare
\let\lb@preparesortA=\lb@preparesort
\def\lb@compare#1\lb@eoe#2\lb@eoe{% SpecialSort:
  \ifx\lb@sorttype\lb@namestring
     \ifx\sortfield\undefined \lb@compareA#1\lb@eoe#2\lb@eoe
     \else 
        \expandafter\RetrieveFieldInFor\expandafter{\sortfield}\lb@entrykey\lb@temp
        \ifx\lb@temp\empty \toks1={#1\lb@eoe}\else \toks1=\expandafter{\lb@temp\lb@eoe}\fi
        \expandafter\RetrieveFieldInFor\expandafter{\sortfield}\lb@currententry\lb@temp
        \ifx\lb@temp\empty \toks2={#2\lb@eoe}\else \toks2=\expandafter{\lb@temp\lb@eoe}\fi
        \edef\lb@temp{\noexpand\lb@compareA\space\the\toks1 \space\the\toks2}\lb@temp
     \fi
  \else \lb@compareA#1\lb@eoe#2\lb@eoe \fi
}
\def\lb@preparesort#1#2\lb@eoe{%
  \if#1-%
    \def\lb@sorttype{#2}%
  \else
    \def\lb@sorttype{#1#2}%
  \fi
  \lb@preparesortA#1#2\lb@eoe
}
\def\SpecialSort#1{\def\sortfield{#1}}
\def\WriteImmediateInfo#1{}  % the existence of .lbr file bocks new reading of .bib
\catcode`\@=\tmpnum

% Main action per every entry:

\def\MakeReference{\par \ifnum\bibnum>0 \bibskip \fi
  \advance\bibnum by1
  \isdefined{bim:\the\bibnum}\iftrue
     \edef\tmpb{\csname bim:\the\bibnum\endcsname}%
     \bibmark=\expandafter{\tmpb}%
  \else \bibmark={}\fi
  \noindent \edef\tmpb{\EntryKey}%
  \printbib \dest[cite:\the\bibnum]%
  \bgroup  
     \RetrieveFieldIn{entrytype}\entrytype
     \csname print:BEGIN\endcsname
     \isdefined{print:\entrytype}\iftrue 
        \csname print:\entrytype\endcsname
     \else 
        \ifx\entrytype\empty \else
           \opwarning{Entrytype @\entrytype\space from [\EntryKey] undefined}%
           \csname print:misc\endcsname
     \fi\fi
     \csname print:END\endcsname
     \ifx\wref\wrefrelax\else \immediate\wref\Xbib{{\EntryKey}{\the\bibnum}{\the\bibmark}}\fi
  \egroup \par
}

% The \bprinta, \bprintb, \bprintc, \bprintv commands used in the style files:

\def\bprinta {\bprintb*}
\def\bprintb #1[#2#3]{%
   \def\bibfieldname{#2#3}%
   \if!#2\relax 
      \def\bibfieldname{#3}%
      \RetrieveFieldIn{#3}\bibfield
      \ifx\bibfield\empty\else
         \RetrieveFieldIn{#3number}\namecount
         \def\bibfield{\csname Read#3\expandafter\endcsname\csname #3name\endcsname}%
      \fi
   \else
      \RetrieveFieldIn{#2#3}\bibfield
   \fi
   \if^#1^%
      \ifx\bibfield\empty \expandafter\expandafter\expandafter \doemptyfield
      \else \expandafter\expandafter\expandafter \dofullfield \fi
   \else \expandafter \bprintaA
   \fi
}
\def\dofullfield#1#2{\def\dofield##1{#1}\expandafter\dofield\expandafter{\bibfield}}
\def\doemptyfield#1#2{\def\dofield##1{#2}\expandafter\dofield\expandafter{\bibfield}}
\let\Readauthor=\ReadAuthor  \let\Readeditor=\ReadEditor
\def\bprintaA #1#2{\ifx\bibfield\empty #2\else\bprintaB #1**\eee\fi}
\def\bprintaB #1*#2*#3\eee{\if^#3^#1\else\expandafter\bprintaC\expandafter{\bibfield}{#1}{#2}\fi}
\def\bprintaC #1#2#3{#2#1#3}
\def\bprintc#1#2{\bprintcA#1#2**\relax}
\def\bprintcA#1#2*#3*#4\relax{\ifx#1\empty \else \if^#4^#2\else#2#1#3\fi\fi}
\def\bprintv [#1]#2#3{\def\tmpa{#2}\def\tmpb{#3}\bprintvA #1,,}
\def\bprintvA #1,{%
   \if^#1^\tmpb\else
      \RetrieveFieldIn{#1}\tmp
      \ifx \tmp\empty 
      \else \tmpa \def\tmpb{}\def\tmpa{}%
      \fi
   \expandafter \bprintvA
   \fi
} 

% Various macros + multilinguas:

\def\bibwarning{\opwarning{Missing field ``\bibfieldname'' in [\EntryKey]}}  

\def\slet#1#2{\expandafter\let\csname#1\expandafter\endcsname\csname#2\endcsname}
\def\mtdef#1#2#3#4{\sdef{mt:#1:en}{#2} \sdef{mt:#1:\czs}{#3}
  \if$#4$\slet{mt:#1:sk}{mt:#1:\czs}
  \else  \sdef{mt:#1:sk}{#4}
  \fi
}
\edef\czs{\csname lan:5\endcsname} % cz (old) or cs (new)

\endinput

%%%%%%%%%%%%%%%%%%%%%%%%%%%%%%%%%%%%%

The following command is implemented here:

   \usebib/<sorttype> (<style>) <bibfiles>

where <sorttype> is one letter 'c' (references ordered by citation order in
the text) or 's' (references ordered by key in the style file), 
<style> is the part of the name `opmac-bib-<style>.tex' of the style file
and <bibfiles> are one or more .bib file names without suffix separated by
comma without space. Example:

   \usebib/s (simple) mybase,yourbase

This command reads the <bibfiles> directly and creates the list of
bibliographics references (only those declared by by \cite[] or \nocite[]
in the text). The formatting of such references is defined in the style
file. The usage is mentioned in OPmac documentation too.

Notes for style writers:
------------------------

The opmac-bib-<style>.tex file must define the commands:

\authorname ... formatting of one name in the authors list. The macro can
    use the following data: \NameCount (the number of the currently
    processed author name in the list), 0\namecount (the total number of the
    authors in the list), \Lastname, \Firstname, \Von, \Junior (the parts of
    the name). See the documentation of the librarian package for more info.

\editorname ... the same as \authorname, but for editors list.

print:<entrytype> (defined by \sdef) for formatting the entry of <entrytype>.
    The <entrytype> have to be lowercase. This command can use the command:

\bprinta [<fieldname>]  {<if defined>} {<if not defined>}. The part <if defined>
    is executed if <fieldname> is declared in .bib file for the entry which is
    currently processed. Else the part <if not defined> is processed.
    The part <if defined> can include the * parameter which is replaced
    by the value of the <fieldname>. The part <if not defined> can include
    the \bibwarning command if the <fieldname> is mandatory.
\bprintb [<fieldname>]  {<if defined>} {<if not defined>}. The same as
    \bprinta, but the ##1 parameter is used instead *. Differences: ##1
    parameter can be used more than once and can be enclosed in nested
    braces. The * parameter can be used at most once and cannot be enclosed
    in braces. Warning: if the \bprintb commands are nested (\bprintb in
    \bprintb), then you need to write ####1 parameter for internal \bprintb.
    But if \bprinta commands are nested then the * parameter is not dubbled.
\pbprintc \macro {<if non-empty>}. The <if non-empty> part is executed if
    \macro is non-empty. The * parameter can be used, it is replaced by
    the \macro.
\bprintv [<field1>,<field2>,...] {<if defined>} {<if not defined>}.
    The part <if defined> is executed if <field1> or <filed2> or ...
    is defined, else the second part <if not defined> is executed.
    There is one filed name or the list field names separated by commas.  
    The parts cannot include any parameter.

There are two special fieldnames: !author and !editor. The processed list of
authors or editors (by repeatedly calling \authorname or \editorname) are 
used here insted of raw data.

You can define print:BEGIN and/or print:END which is executed at the begin or
end of each <entrytype>. The formatting does not solve the numbering and
paragraph indentation of the entry. This is processed by \printbib macro
used in OPmac (and may be redefined by the author or document designer).

You can declare \bimark={something} in the print:END macro. This bibmark is
saved to the .ref file (created by OPmac) and used in the next TeX run as \cite
marks when \nonumcitations is set.

The whole style file is read in the group during \usebib command is executed
before typesetting the reference list. Each definition or setting is local
here.

If you are using non-standard fieldnames in .bib database and bib. style,
you has to declare them by \CreateField {<fieldname>}.

You can declare \SortingOrder in the manner documented by librarian package.

If your style adds some words or abbreviations you can make them
multilingual by saying \mtext{<label>} instead such word and
\mtdef{<label>} {<English>} {<Czech>} {<Slovak>} declaration.
The right part is printed by current value of the \language regiter.
You can add more languages by re-defining the \mtdef command
and by \sdef{lan:<language number>}{<language mark>}. See the opmac.tex
for inspiration.

If you are using \nonumcitations, then the \bibmark toks register have to be
prepared in the style file (in print:BEGIN, print:END, in \authorname etc.)
This value will be used in the \cite[] places in the document.

The example of the style file is in opmac-bib-simple.tex.

User or author of the bib. style can create the hidden field which has a
precedence while sorting names. Example:

   \CreateField {sortedby}
   \SpecialSort {sortedby}

Suppose that the .bib file includes:

   ...
   author   = "Jan Chadima",
   sortedby = "Hzzadima Jan",
   ...

Now, this author is sorted between H and I, because the Ch digraph in this
name has to be sorted by this rule.

If you need (for example) to place the autocitations before other citations,
then you can mark your entries in .bib file by sortedby = "@", because this
character is sorted before A.

%%%%%%%%%%%%%%%%%%%%%%%%%%%%%%%%%%%% end of the opmac-bib.tex
