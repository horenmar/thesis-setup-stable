\input ctustyle
\input pdfuni
\input glosdata
\input opmac-bib
\worktype [B/EN]
\faculty {F3}
\department {Katedra řídící techniky}
\title {The Use of Symbolic Execution for Testing of Real-Time
        Safety-Related Software}
\author {Martin Hořeňovský}
\authorinfo {horenmar@fel.cvut.cz}
\supervisor {Ing. Michal Sojka, Ph.D.}
\studyinfo {Otevřená informatika - Informatika a počítačové vědy}
\date {Květen 2015}
\abstractEN {Safety critical software is hard!.}
\abstractCZ {Safety critical software is hard!
             \rfc{Translate the english abstract}}
\thanks{I would like to thank my supervisor Ing. Michal Sojka, Ph.D.
        for having patience with my lazy ass. \rfc{Selfcensorship?}
        I also want to thank Petr Olšák for creating almost unified template
        for CTU.}
\declaration {
I hereby declare that I made this on my own and I declared
all used sources according to ``Metodický pokyn o dodržování
etických principů při přípravě vysokoškolských závěřečných prací''

\vskip 4cm
\chyph
Prohlašuji, že jsem předloženou práci vypracoval samostatně
a že jsem uvedl veškeré použité informační zdroje v souladu
s Metodickým pokynem o dodržování etických principů při přípravě
vysokoškolských závěrečných prací.

\signature
}


%Some draft iteration related "defines"
\draft %-- Remove this to finalize pdf.
\savetoner
 
\makefront
\chap Introduction

I am citing klee-paper!\cite[klee-paper]
I am citing SymDrive paper!\cite[symdrive-paper]
I am citing mailing list!\cite[wllvm-patch]


\sec Motivation

\sec What is the goal

\sec Structure

\sec Following text



\chap Background and related software

\sec Safety-critical software and security

\sec Symbolic execution

\secc Existing tools

\sec KLEE

KLEE is a symbolic execution tool primarily geared towards performing
high-coverage tests on programs originally created by Cadar et al. \cite[klee-paper]

It is built on the LLVM compiler infrastructure, as it uses its front-end to
convert C code in to LLVM IR representation, which it then works on.

While its original purpose was to test general purpose programs, we \rfc{I?}
decided to test its fitness for testing real-time safety critical software,
because such software's number of paths through code is relatively small.

\secc KLEE integer overflow checking

\sec Real-Time safety-critical applications

\secc eMotor

eMotor is proprietary control software for Infineon Tricore TC1798 motor.

\secc MaCAN

MaCAN library \rfc{cite?} implements MaCAN protocol for Linux, Infineon Tricore TC1798 and STM32 architectures.
MaCAN protocol builds upon CAN protocol to provide additional security through authenticating control messages
received from the network.

\chap Toolchain and case study preparation

\sec KLEE stuff

\rfc{make a better headline}

\sec eMotor modifications

\sec MaCAN modifications

\chap Evaluation

\sec Results - problems, bugs, solutions?

\rfc {Again, better headline} 

\sec Complexity, limitations, execution time

\sec KLEE fitness for purpose

\chap Conclusion



\bibchap
\usebib/c (simple) mybase
%\usebib/<sorttype> (<style>) <bibfile>
%sorttype c -- citation order in text
%sorttype s -- by key in style file


%Supposedly "Zadání práce" if in czech.
\app Specification


\rfc{Include assignment in czech as well.}

\app Glossary\par
\makeglos

\nextoddpage

\bye
